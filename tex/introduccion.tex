% !TEX root = ../ejemplo-memoria.tex
% Contenidos del capítulo.
% Las secciones presentadas son orientativas y no representan
% necesariamente la organización que debe tener este capítulo.

\section{Introducción}

En la actualidad, la tecnología se ha convertido en una herramienta clave para transformar la forma en la que interactuamos con nuestro entorno, permitiendo nuevas maneras de aprender, explorar y conectarnos con la historia y la cultura. Sin embargo, la narrativa histórica y cultural a menudo ha sido contada desde una perspectiva que invisibiliza las contribuciones de las mujeres. Muchas figuras femeninas que desempeñaron papeles fundamentales en la ciencia, el arte, la política y la sociedad han sido relegadas al olvido.

Este Trabajo de Fin de Grado se centra en el desarrollo de una aplicación móvil interactiva basada en geolocalización y realidad aumentada (RA), que tiene como objetivo principal visibilizar el legado de mujeres históricamente olvidadas en la ciudad de Valencia. Mediante esta aplicación, los usuarios podrán desbloquear contenido multimedia y educativo al visitar puntos de interés geolocalizados, conectando los lugares emblemáticos de la ciudad con las historias de estas mujeres, conociendo su vida y obra.

La aplicación, titulada “DONA’m MÓN”, busca no solo recuperar estas narrativas olvidadas, sino también darles un espacio en el mundo mediante una experiencia educativa y lúdica que invite a reflexionar sobre la igualdad de género. Este proyecto pretende ser un puente entre el pasado y el presente, ayudando a los usuarios a redescubrir la ciudad desde esta nueva perspectiva inclusiva.

\section{Motivación}
El desarrollo de este proyecto nace de la necesidad de combatir la invisibilización histórica de las mujeres y de mi interés personal por combinar la tecnología con fines educativos y sociales. Valencia, como muchas otras ciudades, cuenta con una rica historia en la que las contribuciones de las mujeres han sido en gran medida ignoradas o subestimadas. Este proyecto representa una oportunidad para devolverles el reconocimiento que merecen y, al mismo tiempo, aprovechar las posibilidades que la tecnología ofrece para innovar en la forma de transmitir conocimiento histórico.

Mi motivación también está ligada a la idea de ofrecer a la sociedad una herramienta accesible y creativa que permita aprender de manera activa y significativa. En un contexto en el que gran parte de la población cuenta con dispositivos móviles, la integración de tecnologías como la RA y la geolocalización garantiza una experiencia inmersiva y dinámica. Además, al ser un proyecto enfocado en Valencia, me inspira la posibilidad de aportar un valor cultural y educativo a mi entorno local.

En última instancia, este TFG no solo pretende ser un ejercicio técnico, sino también un aporte a la visibilización y la justicia histórica. Creo firmemente que la tecnología puede ser una aliada para transformar las narrativas históricas y sociales, y este proyecto aspira a demostrar cómo es posible unir innovación y compromiso social para crear una herramienta que beneficie a la comunidad.

\section{Objetivos}
El objetivo principal de este proyecto es desarrollar una aplicación móvil interactiva que utilice tecnologías de geolocalización, realidad aumentada (RA) y una base de datos centralizada para visibilizar y divulgar las historias de mujeres destacadas en la ciudad de Valencia, promoviendo su reconocimiento histórico y cultural a través de una experiencia inmersiva, educativa y accesible para los usuarios.

Descomponiendo este objetivo principal obtenemos estos objetivos específicos:

	1.	Investigación y selección de mujeres invisibilizadas:
    
	•	Identificar figuras femeninas históricas relacionadas con Valencia que hayan destacado en campos como la ciencia, el arte, la política o los derechos sociales.
    
	•	Asociar a cada figura un punto geolocalizado en la ciudad con un vínculo significativo con su historia.
    
	2.	Creación y gestión de una base de datos con Django:
    
	•	Diseñar una base de datos relacional para almacenar información detallada sobre las mujeres seleccionadas, los puntos geolocalizados y el contenido multimedia asociado.
    
	•	Implementar un panel de administración con Django que permita gestionar eficientemente los datos de las mujeres, ubicaciones y elementos interactivos de la aplicación.
    
	3.	Desarrollo de funcionalidades tecnológicas:
    
	•	Implementar un sistema de geolocalización que permita a los usuarios desbloquear contenido interactivo al acercarse a los puntos de interés.
    
	•	Integrar elementos de realidad aumentada que permitan visualizar objetos, imágenes o recreaciones 3D relacionados con las mujeres seleccionadas al enfocar con la cámara del dispositivo móvil.
    
	4.	Diseño de una experiencia de usuario intuitiva y atractiva:
    
	•	Crear una interfaz accesible que facilite la navegación y el uso de las funcionalidades principales de la aplicación.
    
	•	Incluir un mapa interactivo con los puntos de interés marcados y accesibles según la ubicación del usuario.
    
	5.	Generación de contenido multimedia:
    
	•	Diseñar materiales interactivos y educativos como textos, audios, imágenes, videos y modelos 3D que permitan a los usuarios conocer la vida y las contribuciones de las mujeres seleccionadas.
    
	•	Desarrollar un sistema de recompensas o logros para incentivar la exploración de todos los puntos geolocalizados.
    
	6.	Compatibilidad multiplataforma y pruebas de usabilidad:
    
	•	Garantizar que la aplicación sea funcional en 
    dispositivos Android e iOS, utilizando herramientas de desarrollo multiplataforma como Unity.
    
	•	Realizar pruebas de usabilidad con un grupo de usuarios para asegurar que la aplicación cumpla con los objetivos educativos y tecnológicos planteados, además de evaluar la experiencia de uso.
    
\section{Organización de la memoria}
