% !TEX root = ../ejemplo-memoria.tex
% Contenidos del capítulo.
% Las secciones presentadas son orientativas y no representan
% necesariamente la organización que debe tener este capítulo.

\section{Introducción}


En la actualidad, la tecnología se ha convertido en una herramienta clave para trans-
formar la forma en la que interactuamos con nuestro entorno, permitiendo nuevas maneras
de aprender, explorar y conectarnos con la historia y la cultura. Sin embargo, la construcción de los relatos históricos y culturales ha tendido a otorgar mayor protagonismo a figuras masculinas, dejando en la sombra numerosas contribuciones femeninas de gran relevancia en ámbitos como la ciencia, el arte, la política o la educación.

Este Trabajo de Fin de Grado se centra en el desarrollo de una aplicación móvil interactiva basada en geolocalización y realidad aumentada (RA), que tiene como objetivo visibilizar el legado de mujeres con relevancia histórica que han nacido, vivido, desarrollado parte de su labor o tenido relevancia en Valencia. Mediante esta aplicación, los usuarios podrán desbloquear contenido educativo y multimedia al visitar puntos de interés geolocalizados, conectando los lugares emblemáticos de la ciudad con las historias de estas mujeres, conociendo su vida y obra.

La aplicación lleva por nombre DONA’m MÓN, una frase en valenciano con doble significado en su primera palabra: “dona” puede leerse tanto como sustantivo (“mujer”) o como forma verbal (“dame”).  Así que el título leído como “Dame mundo”, esconde un juego de palabras que representa el propósito del proyecto: ofrecer a las mujeres este espacio real y simbólico en el mundo, un lugar en el mapa donde su historia pueda ser descubierta y reconocida.

DONA’m MÓN ofrece una experiencia educativa y lúdica que invita a reflexionar sobre la igualdad de género. Este proyecto pretende ser un puente entre el pasado y el presente, ayudando a los usuarios a redescubrir la ciudad desde esta nueva perspectiva inclusiva.



\section{Motivación}
El desarrollo de este proyecto surge de una doble motivación: por un lado, mi interés personal en el desarrollo de tecnologías con un propósito educativo y social, y por otro, el deseo de poner en valor las aportaciones de las mujeres a lo largo de la historia.

Durante mi recorrido académico como estudiante de Ingeniería Multimedia y en mi máster actual en Tecnologías web, Aplicaciones móviles y Computación en la nube, he desarrollado un fuerte interés por aplicar el conocimiento técnico en proyectos que tengan un impacto positivo en la sociedad, orientados a transformar la manera en la que accedemos y transmitimos el conocimiento.

Además, como mujer en STEM, considero importante que este proyecto consiga ayudar en la visibilización del papel femenino en todas las áreas del saber y a toda clase de público. Aunque se han producido avances significativos en materia de igualdad, los datos actuales evidencian que la brecha persiste: solo el 33\% de los investigadores a nivel mundial son mujeres, y en campos como la inteligencia artificial esta cifra desciende al 22\%~\cite{unesco2021}. En el contexto español, apenas el 24\% de las cátedras universitarias están ocupadas por mujeres, según el informe “Científicas en Cifras 2021” del Ministerio de Ciencia e Innovación~\cite{cientificas2021}. Estas cifras reflejan una infrarrepresentación que no solo afecta a la ciencia, sino también a otros ámbitos clave del desarrollo cultural y social.

Valencia, como muchas otras ciudades, alberga una historia rica y compleja en la que también han participado numerosas mujeres valiosas, desde científicas y artistas hasta activistas y educadoras. Muchas de ellas siguen siendo poco conocidas fuera de círculos especializados. Este proyecto representa una oportunidad para darles mayor protagonismo, aprovechando tecnologías como la realidad aumentada y la geolocalización para generar experiencias de aprendizaje activas, accesibles e innovadoras.

Por todo esto, el objetivo de este trabajo no es solo técnico, sino también ético y cultural: contribuir, desde la tecnología, a construir una memoria más justa e inclusiva.

\section{Objetivos}
El objetivo principal de este proyecto es desarrollar una aplicación móvil interactiva que utilice tecnologías de geolocalización, realidad aumentada (RA) y una base de datos centralizada para visibilizar y divulgar las historias de mujeres destacadas en la ciudad de Valencia, promoviendo su reconocimiento histórico y cultural a través de una experiencia inmersiva, educativa y accesible para los usuarios.

Descomponiendo este objetivo principal, se definen los siguientes objetivos específicos:

\begin{itemize}
    \item \textbf{Investigación y selección de mujeres invisibilizadas:}
    \begin{itemize}
        \item Identificar figuras femeninas históricas relacionadas con Valencia que hayan destacado en campos como la ciencia, el arte, la política o los derechos sociales.
        \item Asociar a cada figura un punto geolocalizado en la ciudad con un vínculo significativo con su historia.
    \end{itemize}

    \item \textbf{Creación y gestión de una base de datos con MySQL:}
    \begin{itemize}
        \item Diseñar y desarrollar una base de datos relacional en MySQL, estructurada para almacenar de forma eficiente información sobre las mujeres seleccionadas, sus ubicaciones geográficas y el contenido multimedia vinculado.
        \item Implementar un backend utilizando el framework Django, asegurando la integridad, consistencia y seguridad de los datos mediante su ORM, y exponiendo una API RESTful que permita su acceso y manipulación desde la aplicación móvil.
    \end{itemize}


    \item \textbf{Desarrollo de funcionalidades tecnológicas:}
    \begin{itemize}
        \item Implementar un sistema de geolocalización que permita a los usuarios desbloquear contenido interactivo al acercarse a los puntos de interés.
        \item Integrar elementos de realidad aumentada desarrollados con A-Frame, que permitan visualizar objetos, imágenes o recreaciones 3D relacionadas con las mujeres seleccionadas al enfocar con la cámara del dispositivo móvil.
        \item Implementar un sistema de notificaciones que alerte a los usuarios al aproximarse a un punto de interés relevante, mejorando la interacción con el entorno urbano y fomentando la exploración activa.
    \end{itemize}

    \item \textbf{Diseño de una experiencia de usuario intuitiva y atractiva:}
    \begin{itemize}
        \item Crear una interfaz accesible y adaptada al entorno móvil con React Native y Expo Go, que facilite la navegación y el uso de las funcionalidades principales de la aplicación.
        \item Incluir un mapa interactivo con los puntos de interés marcados, accesibles según la ubicación del usuario.
    \end{itemize}

    \item \textbf{Generación de contenido multimedia:}
    \begin{itemize}
        \item Diseñar materiales interactivos y educativos como textos, audios, imágenes, vídeos y modelos 3D que permitan a los usuarios conocer la vida y las contribuciones de las mujeres seleccionadas.
        \item Desarrollar un sistema de recompensas o logros para incentivar la exploración de todos los puntos geolocalizados.
    \end{itemize}

    \item \textbf{Compatibilidad multiplataforma y pruebas de usabilidad:}
    \begin{itemize}
        \item Garantizar que la aplicación sea funcional en dispositivos Android e iOS mediante el uso de herramientas multiplataforma como React Native y Expo Go.
        \item Realizar pruebas de usabilidad con un grupo de usuarios para evaluar la experiencia de uso y verificar que se cumplen los objetivos educativos y tecnológicos planteados.
    \end{itemize}
\end{itemize}

\subsection{Alineación con los Objetivos de Desarrollo Sostenible (ODS)}

DONA'm MÓN se relaciona de forma directa con algunos de los Objetivos de Desarrollo Sostenible propuestos por la ONU dentro de la Agenda 2030, especialmente aquellos centrados en la igualdad, la educación, la sostenibilidad urbana y el bienestar. A través de una experiencia tecnológica que combina cultura, memoria y participación ciudadana, el proyecto contribuye a acercar estos objetivos al contexto local. \cite{onuAgenda2030,odsIgualdad,odsCiudades}

\begin{itemize}
    \item \textbf{ODS 5: Igualdad de género.} Este objetivo busca eliminar todas las formas de discriminación y violencia contra mujeres y niñas, y promover su plena participación en todos los ámbitos. DONA'm MÓN se centra precisamente en recuperar y visibilizar figuras femeninas que han sido olvidadas o ignoradas. Reivindicar sus historias es una forma de justicia simbólica y de construir referentes para las nuevas generaciones.

    \item \textbf{ODS 4: Educación de calidad.} Este ODS promueve el acceso a una educación inclusiva, equitativa y de calidad. La aplicación funciona como una herramienta educativa alternativa que permite aprender fuera del aula, en el espacio urbano, utilizando recursos multimedia y tecnologías actuales. Es una forma distinta de acercarse a la historia local desde una perspectiva crítica y feminista.

    \item \textbf{ODS 11: Ciudades y comunidades sostenibles.} Uno de los enfoques de este objetivo es garantizar que las ciudades sean más inclusivas y accesibles. DONA'm MÓN propone una forma de recorrer Valencia nueva, resignificando lugares que muchas veces pasan desapercibidos. También invita a caminar y descubrir el entorno, fomentando un uso más consciente del espacio público.

    \item \textbf{ODS 3: Salud y bienestar.} Aunque no es el objetivo principal del proyecto, se puede decir que la aplicación promueve indirectamente hábitos saludables, ya que para desbloquear los contenidos de las rutas es necesario moverse físicamente por la ciudad. Esto puede motivar a las personas usuarias a caminar, explorar y mantenerse activas mientras aprenden.
\end{itemize}

    
\section{Organización de la memoria}
Esta memoria se estructura en siete capítulos principales, además de un apéndice y una bibliografía final. A continuación se describen brevemente los contenidos de cada uno de ellos:

Capítulo 1 – Introducción: contextualiza el trabajo realizado, presentando la motivación personal y social del proyecto, así como los objetivos generales y específicos que se pretenden alcanzar. También incluye esta sección de organización de la memoria para orientar al lector sobre la estructura del documento.

Capítulo 2 – Estado del arte: recoge un análisis comparativo de aplicaciones similares ya existentes, con el objetivo de identificar puntos fuertes y carencias que puedan aprovecharse como referencia para el desarrollo del proyecto. Además, se detallan las tecnologías empleadas, justificando su elección en función de las necesidades de la aplicación.

Capítulo 3 – Requisitos, especificaciones, costes, riesgos y viabilidad: define los requisitos funcionales y no funcionales del sistema, junto con sus especificaciones técnicas. También se lleva a cabo una estimación de los costes asociados, se identifican posibles riesgos y se valora la viabilidad del proyecto desde una perspectiva técnica y económica.

Capítulo 4 – Análisis: describe el análisis de la solución propuesta, incluyendo aspectos como la estructura general de la aplicación, la lógica funcional y la interacción entre sus componentes principales.

Capítulo 5 – Diseño: presenta el diseño de la aplicación a diferentes niveles (arquitectura, interfaz de usuario, estructura de datos, etc.), sirviendo como puente entre el análisis conceptual y la posterior implementación técnica.

Capítulo 6 – Implementación y pruebas: documenta el proceso de desarrollo, indicando cómo se ha llevado a cabo la implementación de la solución y qué herramientas se han utilizado. A su vez, se incluyen las pruebas realizadas para verificar el correcto funcionamiento del sistema, abarcando pruebas funcionales, de rendimiento y de usabilidad.

Capítulo 7 – Conclusiones: ofrece una reflexión final sobre los resultados obtenidos, revisa los costes reales frente a los estimados, valora el cumplimiento de los objetivos y propone posibles mejoras o líneas de trabajo futuro.

Bibliografía: recoge todas las fuentes utilizadas para la elaboración del trabajo, tanto en su fase de investigación como en el desarrollo e implementación del proyecto.

Esta organización busca facilitar la lectura y comprensión del documento, guiando al lector a través del proceso completo de conceptualización, desarrollo y evaluación de la aplicación DONA’m MÓN.